\documentclass{beamer}
\usepackage[utf8]{inputenc}

\begin{document}

\title{Summary of Polson and Sokolov 2018}
\subtitle{Deep Learning for Energy Markets}
\author{David Prentiss}
\institute{OR750-004}
\date{\today}

\frame{\titlepage}

\begin{frame}
  \frametitle{Recurrent neural network}
\end{frame}

\begin{frame}
  \frametitle{Recurrent neural network}
\end{frame}

\begin{frame}
  \frametitle{Long short-term memory}
  Vanilla RNN
  \begin{equation*}
    h_t=\tanh\left( W
      \begin{pmatrix}
        h_{t-1} \\ x_t
      \end{pmatrix}
    \right)
  \end{equation*}
  LSTM
  \begin{align*}
    \begin{pmatrix}
      i \\ f \\ o \\ k
    \end{pmatrix}
    &=
    \begin{pmatrix}
      \sigma \\ \sigma \\ \sigma \\ \tanh
    \end{pmatrix}
    \circ
    W
    \begin{pmatrix}
      h_{t-1} \\ x_t
    \end{pmatrix}
    \\
    c_t &= f \odot c_{t-1} + i \odot k \\
    h_t &= o \odot \tanh\left(c_t\right)
  \end{align*}
\end{frame}

\begin{frame}
  \frametitle{Extreme value theory}
  \begin{itemize}
  \item Extreme value analysis begins by filtering the data to select
    ``extreme'' values.
    \item Extreme values are selected by one of two methods.
  \begin{itemize}
  \item Block maxima: Select the peak values after dividing the series into periods.
  \item Peak over threshold: Select values larger than some threshold.
  \end{itemize}
    \item Peak over threshold used in this paper.
  \end{itemize}
\end{frame}

\begin{frame}
  \frametitle{Peak over threshold}
\end{frame}

\begin{frame}
  \frametitle{Generalized Pareto distribution}
  \begin{itemize}
  \item CDF
    \begin{equation*}
      H(y\mid\sigma,\xi)
      =
      1-\left(1+\xi\frac{y-u}{\sigma}\right)_+^{\frac{-1}{\xi}}
    \end{equation*}
  \item PDF
    \begin{equation*}
      h(y\mid\sigma,\xi)
      =
      1-\frac{1}{\sigma}\left(1+\xi\frac{y-u}{\sigma}\right)^{\frac{-1}{\xi}-1}
    \end{equation*}
  \end{itemize}

\end{frame}

\begin{frame}
  \frametitle{Parameters}
  \begin{itemize}
  \item Location, \(u\), is the threshold
  \item Scale, \(\sigma\), is our learned parameter
  \item Shape, \(\xi = f(u, \sigma)\)?
    \(\text{EX}\left[y\right] = \sigma+u \implies \xi = 0\)?

  \end{itemize}

\end{frame}

\end{document}